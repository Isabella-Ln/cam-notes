% header.tex -- common preamble for IB notes (Linear Algebra etc.)

% 基本數學套件
\usepackage{amsmath,amssymb,amsthm}
\usepackage{mathtools}
\usepackage{bm}          % \bm{v} 粗體向量
\usepackage{tikz-cd}     % commutative diagrams

% 版面(可依個人喜好調整)
\usepackage[a4paper,margin=2.5cm]{geometry}
\usepackage{enumitem}
\setlist[enumerate]{label=(\roman*),wide}

% 數域簡寫
\newcommand{\R}{\mathbb{R}}
\newcommand{\C}{\mathbb{C}}
\newcommand{\F}{\mathbb{F}}

% 常用算子
\DeclareMathOperator{\supp}{supp}
\DeclareMathOperator{\im}{im}
\DeclareMathOperator{\Mat}{Mat}
\DeclareMathOperator{\GL}{GL}
\DeclareMathOperator{\id}{id}

% Dirac 風格括號
\newcommand{\bra}{\langle}
\newcommand{\ket}{\rangle}

% 額外:rank / nullity 如有需要
\DeclareMathOperator{\rank}{rank}
\DeclareMathOperator{\nullity}{nullity}

% 式子編號跟 section 綁在一起
\numberwithin{equation}{section}

% 定理環境設定
\theoremstyle{plain}
\newtheorem{thm}{Theorem}[section]
\newtheorem{lemma}[thm]{Lemma}
\newtheorem{prop}[thm]{Proposition}
\newtheorem{cor}[thm]{Corollary}

\theoremstyle{definition}
\newtheorem{defi}[thm]{Definition}
\newtheorem{eg}[thm]{Example}
\newtheorem{notation}[thm]{Notation}

\theoremstyle{remark}
\newtheorem*{remark}{Remark}

% 允許比較寬的矩陣
\setcounter{MaxMatrixCols}{20}
